% Created 2015-05-02 土 19:07
\documentclass[presentation]{beamer}
\usepackage[utf8]{inputenc}
\usepackage[T1]{fontenc}
\usepackage{fixltx2e}
\usepackage{graphicx}
\usepackage{longtable}
\usepackage{float}
\usepackage{wrapfig}
\usepackage{rotating}
\usepackage[normalem]{ulem}
\usepackage{amsmath}
\usepackage{textcomp}
\usepackage{marvosym}
\usepackage{wasysym}
\usepackage{amssymb}
\usepackage{hyperref}
\tolerance=1000
\usetheme{default}
\author{}
\date{\today}
\title{EmacsでWordやPowerPoint資料を作成してみる(beamer編)}
\begin{document}

\maketitle
前回の続きです.
\begin{itemize}
\item \href{http://futurismo.biz/archives/3601}{Emacs org-mode でPowerPoint資料を作成してみる(ODT経由)(失敗) | Futurismo}
\end{itemize}

前回は、ODT経由でPowerPoint資料をつくろうとして、失敗しました.

今回は、Beamer(Latex)経由で作成してみます.

\begin{frame}[fragile,label=sec-1]{org-mode形式 -> beamer形式 -> PDF形式 -> ODP形式 -> PPT形式}
 ox-beamerパッケージを利用する. これは、org-modeにデフォルトで入っているので、
以下のようにrequireする.

\begin{verbatim}
(require 'ox-beamer)
\end{verbatim}

実行のためには、 \texttt{pdflatex} コマンドが必要.
Latex環境をまるまるインストールしてしましった.
\begin{itemize}
\item \href{http://did2memo.net/2014/03/06/easy-latex-install-windows-8-2014-03/}{簡単\LaTeX{}インストールWindows編(2014年7月版)}
\end{itemize}

M-x org-beamer-export-to-pdf を実行すると、
beamerをPDFに変換したものを出力できる.



\begin{block}{Special Thanks}
この動画に惹かれた.
\end{block}
\end{frame}
% Emacs 25.0.50.1 (Org mode 8.3beta)
\end{document}
